\documentclass[spanish]{article}
\usepackage[spanish]{babel}
\usepackage[fontsize=14pt]{scrextend}
\usepackage[letterpaper,left=1.5cm,top=2.0cm,bottom=2.0cm,right=2.0cm ]{geometry}
\usepackage[utf8]{inputenc}
\usepackage[T1]{fontenc}

\usepackage{amsmath,amssymb}
\usepackage{graphicx}
\usepackage{tikz}
\usetikzlibrary{shapes}
\usepackage{natbib}
\usepackage{hyperref}
\usepackage{titling}

\title{APUNTES DE QUÍMICA \vspace{-1em}}

\author{Osvaldo Hernández Morales}

\date{}

\setlength{\parindent}{4em}
\setlength{\parskip}{1em}
\setlength{\droptitle}{-5em}  

\begin{document}


\maketitle
\vspace{-5em}


\section*{TABLA PERIÓDICA}
La tabla periódica de los elementos es una disposición de los elementos químicos en forma de tabla, ordenados por su número atómico (número de protones), por su configuración de electrones y sus propiedades químicas. Este ordenamiento muestra tendencias periódicas, como elementos con comportamiento similar en la misma columna.

Las filas de la tabla se denominan períodos y las columnas grupos. Algunos grupos tienen nombres, así por ejemplo el grupo 17 es el de los halógenos y el grupo 18 el de los gases nobles. La tabla también se divide en cuatro bloques con algunas propiedades químicas similares. Debido a que las posiciones están ordenadas, se puede utilizar la tabla para obtener relaciones entre las propiedades de los elementos, o pronosticar propiedades de elementos nuevos todavía no descubiertos o sintetizados. La tabla periódica proporciona un marco útil para analizar el comportamiento químico y es ampliamente utilizada en química y otras ciencias.

\subsection*{HISTORIA}

En el siglo XIX, cuando los químicos sólo tenían una vaga idea respecto de los átomos y las  moléculas, y sin saber aún de la existencia de los electrones y protones, desarrollaron una tabla periódica utilizando su conocimiento de las masas atómicas. Ya se habían hecho mediciones exactas de la masa atómica de muchos elementos. Ordenar los elementos de acuerdo con sus masas atómicas en una tabla periódica parecía una idea lógica para los químicos de aquella época, quienes pensaban que el comportamiento químico debería estar relacionado, de alguna
manera, con las masas atómicas.

En 1864, el químico inglés John Newlands observó que cuando los elementos se ordenaban según sus masas atómicas, cada octavo elemento mostraba propiedades semejantes. Newlands se refirió a esta peculiar relación como la ley de las octavas. Sin embargo, tal “ley” resultó inadecuada para elementos de mayor masa que el calcio, por lo cual el trabajo de Newlands fue rechazado por la comunidad científica.

En 1869, el químico ruso Dmitri Mendeleev y el químico alemán Lothar Meyer propusieron de manera independiente una disposición en tablas mucho más amplia para los elementos, basada en la  repetición periódica y regular de sus propiedades. El sistema de clasificación de Mendeleev superó sobremanera al de Newlands, en particular en dos aspectos. Primero, agrupó los elementos en forma más exacta, de acuerdo con sus propiedades, y segundo, porque hizo viable la predicción de las propiedades de varios elementos que aún no se descubrían.

Por ejemplo, Mendeleev planteó la existencia de un elemento desconocido que llamó ekaaluminio y predijo algunas de sus propiedades (eka es una palabra en sánscrito que significa
“primero”; así, el eka-aluminio sería el primer elemento bajo el aluminio en el mismo grupo).


Los mejores apuntes weee-.



Testing some citations: \cite{Lim2016TEI}


\bibliographystyle{unsrt}
\bibliography{references}


\end{document}